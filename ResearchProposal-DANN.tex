\documentclass[12pt, a4paper]{article}
\usepackage[utf8]{inputenc}
\usepackage[margin=0.5in]{geometry}
\usepackage{amsmath, amsfonts, amssymb, amsthm}
\usepackage{algorithm}
\usepackage{algpseudocode}
\usepackage{graphicx}
\usepackage{hyperref}
\usepackage{enumitem}
\usepackage{abstract}
\usepackage{titlesec}
\usepackage{cite}

% Define operators
\DeclareMathOperator*{\argmax}{arg\,max}
\DeclareMathOperator{\dist}{dist}

% Theorems and definitions
\newtheorem{definition}{Definition}
\newtheorem{theorem}{Theorem}
\newtheorem{lemma}{Lemma}
\newtheorem{conjecture}{Conjecture}
\newtheorem{property}{Property}

% Title formatting
\titleformat{\section}{\normalfont\Large\bfseries}{\thesection}{1em}{}
\titleformat{\subsection}{\normalfont\large\bfseries}{\thesubsection}{1em}{}

% Spacing
\setlength{\parskip}{1em}
\setlength{\parindent}{0em}
\setlength{\absleftindent}{0mm}
\setlength{\absrightindent}{0mm}

\title{\vspace{-2cm}\textbf{Decentralized Autonomous Narrative Networks (DANN): \\ A Multi-Model Approach to Reinforcement Learning}}
\author{\textbf{Paul Lowndes} \\ \href{mailto:ZeroTrust@NSHkr.com}{\texttt{ZeroTrust@NSHkr.com}}}
\date{\small January 1, 2025}

\begin{document}

\maketitle
\vspace{-1.5em}

\begin{abstract}
[Previous abstract remains the same]
\end{abstract}

\section{Introduction}
The proliferation of artificial intelligence systems and their increasing role in shaping information flows has created new challenges in understanding and managing narrative dynamics in digital spaces. Traditional multi-agent reinforcement learning (MARL) approaches often fail to capture the nuanced interplay between agents' beliefs, knowledge, and the narratives they construct and propagate. This paper introduces Decentralized Autonomous Narrative Networks (DANN), a framework that explicitly models these dynamics through a combination of embedding spaces, belief systems, and narrative evolution mechanisms.

\section{Framework Overview}
\subsection{Core Components}
We begin by defining the fundamental mathematical structures that underpin the DANN framework:

\begin{definition}[Embedding Space]
The global embedding space $E_G$ is a metric space $(E_G, d)$ where:
\begin{itemize}
    \item $d: E_G \times E_G \rightarrow \mathbb{R}_{\geq 0}$ is a distance function
    \item For all $x,y \in E_G$: $d(x,y) = 0 \iff x = y$ (identity)
    \item For all $x,y \in E_G$: $d(x,y) = d(y,x)$ (symmetry)
    \item For all $x,y,z \in E_G$: $d(x,z) \leq d(x,y) + d(y,z)$ (triangle inequality)
\end{itemize}
\end{definition}

\begin{definition}[Agent Space]
For each agent $a_i$, its local embedding space $E_i \subseteq E_G$ is equipped with:
\begin{itemize}
    \item Knowledge set $K_{i,t} \subset E_i$ at time $t$
    \item Belief set $B_{i,t} \subset E_i$ at time $t$
    \item Narrative sequence $N_{i,t} = (c_{i,1}, c_{i,2}, \dots, c_{i,T}) \in E_i^T$
\end{itemize}
where $K_{i,t} \subseteq B_{i,t}$ (knowledge is a subset of beliefs).
\end{definition}

\section{Mathematical Framework}

\subsection{Veracity Function Properties}
The veracity function $V: E_G \rightarrow [0,1]$ satisfies:

\begin{property}[Veracity Axioms]
For all $x,y \in E_G$:
\begin{itemize}
    \item $V(x) = 1 \iff x \in T$ (truth region)
    \item $\|x-y\| \leq \epsilon \implies |V(x) - V(y)| \leq \delta$ (continuity)
    \item $V(x) = 0 \implies x$ is maximally inconsistent with truth
\end{itemize}
\end{property}

\subsection{Narrative Dynamics}

\begin{definition}[Narrative Divergence]
The divergence $D$ between narratives satisfies:
\begin{equation}
    D(N_{i,t}, N_{j,t}) = \sum_{k=1}^T w(c_{i,k}) \cdot d(c_{i,k}, c_{j,k})
\end{equation}
where $w(c) = f(V(c))$ for some monotonic function $f:[0,1] \rightarrow [0,1]$.
\end{definition}

\subsection{Agent Interaction Mechanisms}

\subsubsection{Knowledge Propagation}
Knowledge updates follow:
\begin{equation}
    K_{i,t+1} = K_{i,t} \cup \{e \in E_i \mid V(e, T) > \tau_K \land \exists j: e \in K_{j,t}\}
\end{equation}
where $\tau_K$ is the knowledge acceptance threshold.

\subsubsection{Belief Evolution}
Belief updates incorporate both knowledge and social influence:
\begin{equation}
    B_{i,t+1} = f_B(B_{i,t}, K_{i,t+1}, \sum_{j \neq i} \alpha_{ij} B_{j,t})
\end{equation}
where $\alpha_{ij}$ represents the influence weight of agent $j$ on agent $i$.

\section{Learning Mechanisms}

\subsection{Narrative-Based Reward}
The reward function combines environmental and narrative quality:
\begin{equation}
    R_i(s_t, a_t, s_{t+1}) = \alpha \cdot R_{\text{env}}(s_t, a_t, s_{t+1}) + \beta \cdot Q(N_{i,t+1})
\end{equation}
where:
\begin{itemize}
    \item $Q(N) = \gamma_1 C(N) + \gamma_2 V_{\text{avg}}(N) + \gamma_3 I(N)$
    \item $C(N)$ measures narrative coherence
    \item $V_{\text{avg}}(N)$ is the average veracity
    \item $I(N)$ measures narrative influence
\end{itemize}

\subsection{Agent-Switching Mechanism}
The switching function is defined as:
\begin{equation}
    S_i(t) = \argmax_{j} \{Q(M_{i,j}, N_{i,t}, \text{Context}_t) + \lambda H(j)\}
\end{equation}
where:
\begin{itemize}
    \item $H(j)$ is an entropy term promoting exploration
    \item $\lambda$ balances exploitation vs. exploration
    \item $\text{Context}_t$ includes environmental and social factors
\end{itemize}

\section{Discussion and Future Work}
[This section would discuss implications, limitations, and future research directions]

\end{document}